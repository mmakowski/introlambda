\documentclass[11pt,twoside,a4paper]{article} % scrartcl

\usepackage{amsmath}
\usepackage[hidelinks]{hyperref}
\usepackage{mdframed}
\usepackage{syntax}

\begin{document}
\title{Introduction to Lambda Calculus}
%\subtitle{notes}
\author{Maciek Makowski}
\maketitle

\section{Motivation}

TODO

\section{Syntax}

\setlength{\grammarindent}{6em} 
\begin{grammar}
<term> ::= $v$ 
   \alt $\lambda v.$<term> 
   \alt <term> <term>
\end{grammar}
where $v\in V$, a set of variable names.

Examples: $v_1$, $x\ y$, $(\lambda a.\lambda b.a)\ c\ (\lambda a.b)$

The rule with 

\section{Rewriting Rules}

The grammar tells us how to generate arbitrary lambda-terms. We can define the
following operations on those terms:
\begin{itemize}
\item renaming of bound variables (\emph{$\alpha$-conversion}); e.g. $(\lambda
x.x\,y)\ (\lambda x.x)\longleftrightarrow_\alpha(\lambda
a.a\,y)\ (\lambda b.b)$
\item removal of abstraction under application (\emph{$\beta$-reduction}); e.g. $(\lambda
x.x\,y)\ (\lambda x.x)\longrightarrow_\beta y$
\item introduction/removal of redundant abstraction (\emph{$\eta$-conversion});
e.g. $\lambda x.y\,x\longleftrightarrow_\eta y$
\end{itemize}
TODO: note on subtleties in substitution

TODO: expand on each rule

\section{Foundational Theory}

What exactly is the number 2? Foundational theories of mathematics attempt to 
provide a concrete answer in terms of some primitive objects. The best known 
example is that based in set theory. Using Peano's arithmetic where $2=S(S(0))$
($S$ is the successor function) natural numbers can be modelled as follows:
\begin{align*}
0 &= \emptyset \\
1 &= \left\{\emptyset\right\} \\
2 &= \left\{\left\{\emptyset\right\}, \emptyset\right\}
\end{align*}

\begin{mdframed}
In general, $S(n) = n \cup \left\{n\right\}$.
\end{mdframed}

It turns out that all known mathematical concepts can be stated in terms of 
set theory. Lambda calculus was conceived by Church as an alternative 
foundational theory in which mathematics can be embedded\cite{grue97}. 
How do we model natural numbers in it?
\begin{align*}
0 &= \lambda s.\lambda z.z \\
1 &= \lambda s.\lambda z.s\,z \\
2 &= \lambda s.\lambda z.s\,(s\,z)
\end{align*}

\begin{mdframed}
In general, $S(n) = \lambda s.\lambda z\underbrace{s\,(\dots s\,(s}_n\,z)\dots)$.
\end{mdframed}

\section{Model of Computation}

We have seen that simple arithmetic operations can be expressed in 
TODO

\section{Programming Language}

TODO

\section{Church-Rosser Theorem}

TODO

\section{Curry-Howard Correspondence}

TODO

\section{Further Reading}

A direct inspiration for this talk was the presentation of lambda calculus in
\cite{TAPL}. The book is very well written and builds a sophisticated type
system in easy to follow steps, starting from untyped lambda calculus. 

For a succinct but rigorous introduction to lambda calculus see \cite{bb00}.

TODO

\begin{thebibliography}{9}
\bibitem{TAPL} Benjamin C. Pierce, \emph{Types and Programming Languages},
\url{http://www.cis.upenn.edu/~bcpierce/tapl/}
\bibitem{grue97} Klaus Grue, \emph{Lambda calculus as a foundation of mathematics},
\url{http://www.diku.dk/~grue/papers/church/church.html}
\bibitem{bb00} Henk Berendregt, Erik Barendsen, \emph{Introduction to Lambda
Calculus}, \url{http://www.cse.chalmers.se/research/group/logic/TypesSS05/Extra/geuvers.pdf}
\end{thebibliography}

\end{document}
